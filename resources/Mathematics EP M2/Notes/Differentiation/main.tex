\documentclass{fts_m2}
\title{Differentiation}
\begin{document}
    \maketitle\clearpage
    \section*{Introduction}
    What is differentiation? Broadly speaking, differentiation is the study of the so-called \textit{linear approximation} of a function, or, more appetizingly, the tendency of the trend that some function $f(t)$ will take when $t$ increases.\\
    The studies of approximation, trends and optima are long-standing problems of many aspects of development like engineering.
    The first sight of differentiation can even be found in ancient Greek, Japan and China.\\
    This note is divided into 3 parts.
    The first two sections are purely about the intuition behind the definition of derivative (the first principle), for those who think that this definition is weird.
    From the third section onwards, we focus on the techniques and the laws that helps us to differentiate a function without having to attempt the first principle everytime.
    This includes product rule, quotient rule, chain rule, etc..
    The last few sections are concerning the application of differentiation to different problems, which is a heat topic in exams.\\
    Sometimes there are some supplementary information at the end of a section. All of them are skippable.\\
    Throughout this note, unless otherwise specified, we are interested in functions $I\to\mathbb R$, where $I$ is a union of some open intervals in $\mathbb R$.

    \section{Intuition}
    We already know that the derivative of the exponential function is itself.
That is
$$\frac{\mathrm de^x}{\mathrm dx}=e^x$$
It is natual for us to enquire the case for the opposite: the logarithm.\\
There is nothing very polished here, just the usual first principle.
\begin{align*}
    \lim_{h\to0}\frac{\ln(x+h)-\ln(x)}{h}
    =&\lim_{h\to0}\ln\left(\left(1+\frac{1}{x}h\right)^{1/h}\right)\\
    =&\ln\left(\lim_{h\to0}\left(1+\frac{1}{x}h\right)^{1/h}\right)\\
    =&\ln(e^{1/x})\\
    =&\frac{1}{x}
\end{align*}
Therefore the logarithm function is differentiable and its derivative is $1/x$.
    \subsection*{Exercises}
    
  
    
    \section{The first principle}
    We already know that the derivative of the exponential function is itself.
That is
$$\frac{\mathrm de^x}{\mathrm dx}=e^x$$
It is natual for us to enquire the case for the opposite: the logarithm.\\
There is nothing very polished here, just the usual first principle.
\begin{align*}
    \lim_{h\to0}\frac{\ln(x+h)-\ln(x)}{h}
    =&\lim_{h\to0}\ln\left(\left(1+\frac{1}{x}h\right)^{1/h}\right)\\
    =&\ln\left(\lim_{h\to0}\left(1+\frac{1}{x}h\right)^{1/h}\right)\\
    =&\ln(e^{1/x})\\
    =&\frac{1}{x}
\end{align*}
Therefore the logarithm function is differentiable and its derivative is $1/x$.
    \subsection*{Exercises}
    

    \subsection*{Supplementary Information}
    The derivative of polynomials is not merely a tool we used to analysie the groowth of polynomial functions, but also their algebraic properties.\\
For example,
\begin{definition}
    Let $r$ be a real number and $P(x)$ a nonconstant polynomial.\\
    If $r$ is a root of $P(x)$, then it is said to be with multiplicity $m$ if and only if
    $$P(x)=(x-r)^mQ(x)$$
    where $Q(r)\neq 0$ (that is, $r$ is not a root of $Q$).\\
    Otherwise, we define the multiplicity of $r$ in $P$ to be $0$. 
\end{definition}
\begin{remark}
    It is obvious that the multiplicity is well-defined. It is, indeed, the minimal $m$ such that $(x-r)^m$ divides $P$.
\end{remark}
\begin{theorem}
    Let $P(x)$ be a nonconstant polynomial and $r$ a root of it with multiplicity $n$. Then the multiplicity of $r$ in $P^\prime(x)$ is $n-1$.
\end{theorem}
\begin{proof}
    Suppose that $P(x)=(x-r)^nQ(x)$, then
    \begin{align*}
        P^\prime(x)=&n(x-r)^{n-1}Q(x)+(x-r)^nQ^\prime(x)\\
        =&(x-r)^{n-1}(nQ(x)+(x-r)Q^\prime(x))
    \end{align*}
    Now, since $Q(r)\neq 0$, $r$ is not a root of $nQ(x)+(x-r)Q^\prime(x)$ (simply plug it in). This means that $r$ has multiplicity $n-1$ in $P^\prime(x)$.
\end{proof}
There are many other interesting algebraic or even number theorectical results relevant to the derivative of a polynomial.
The Hensel's lemma, for example, might be one of the most fundamental results in building the $p$-adic number theory.

    \section{Differentiating Sums, Products and Quotients}
    We already know that the derivative of the exponential function is itself.
That is
$$\frac{\mathrm de^x}{\mathrm dx}=e^x$$
It is natual for us to enquire the case for the opposite: the logarithm.\\
There is nothing very polished here, just the usual first principle.
\begin{align*}
    \lim_{h\to0}\frac{\ln(x+h)-\ln(x)}{h}
    =&\lim_{h\to0}\ln\left(\left(1+\frac{1}{x}h\right)^{1/h}\right)\\
    =&\ln\left(\lim_{h\to0}\left(1+\frac{1}{x}h\right)^{1/h}\right)\\
    =&\ln(e^{1/x})\\
    =&\frac{1}{x}
\end{align*}
Therefore the logarithm function is differentiable and its derivative is $1/x$.
    \subsection*{Exercises}
    

    \subsection*{Supplementary Information}
    The derivative of polynomials is not merely a tool we used to analysie the groowth of polynomial functions, but also their algebraic properties.\\
For example,
\begin{definition}
    Let $r$ be a real number and $P(x)$ a nonconstant polynomial.\\
    If $r$ is a root of $P(x)$, then it is said to be with multiplicity $m$ if and only if
    $$P(x)=(x-r)^mQ(x)$$
    where $Q(r)\neq 0$ (that is, $r$ is not a root of $Q$).\\
    Otherwise, we define the multiplicity of $r$ in $P$ to be $0$. 
\end{definition}
\begin{remark}
    It is obvious that the multiplicity is well-defined. It is, indeed, the minimal $m$ such that $(x-r)^m$ divides $P$.
\end{remark}
\begin{theorem}
    Let $P(x)$ be a nonconstant polynomial and $r$ a root of it with multiplicity $n$. Then the multiplicity of $r$ in $P^\prime(x)$ is $n-1$.
\end{theorem}
\begin{proof}
    Suppose that $P(x)=(x-r)^nQ(x)$, then
    \begin{align*}
        P^\prime(x)=&n(x-r)^{n-1}Q(x)+(x-r)^nQ^\prime(x)\\
        =&(x-r)^{n-1}(nQ(x)+(x-r)Q^\prime(x))
    \end{align*}
    Now, since $Q(r)\neq 0$, $r$ is not a root of $nQ(x)+(x-r)Q^\prime(x)$ (simply plug it in). This means that $r$ has multiplicity $n-1$ in $P^\prime(x)$.
\end{proof}
There are many other interesting algebraic or even number theorectical results relevant to the derivative of a polynomial.
The Hensel's lemma, for example, might be one of the most fundamental results in building the $p$-adic number theory.

    \section{Derivatives of Integral Powers}
    We already know that the derivative of the exponential function is itself.
That is
$$\frac{\mathrm de^x}{\mathrm dx}=e^x$$
It is natual for us to enquire the case for the opposite: the logarithm.\\
There is nothing very polished here, just the usual first principle.
\begin{align*}
    \lim_{h\to0}\frac{\ln(x+h)-\ln(x)}{h}
    =&\lim_{h\to0}\ln\left(\left(1+\frac{1}{x}h\right)^{1/h}\right)\\
    =&\ln\left(\lim_{h\to0}\left(1+\frac{1}{x}h\right)^{1/h}\right)\\
    =&\ln(e^{1/x})\\
    =&\frac{1}{x}
\end{align*}
Therefore the logarithm function is differentiable and its derivative is $1/x$.
    \subsection*{Supplementary Information}
    The derivative of polynomials is not merely a tool we used to analysie the groowth of polynomial functions, but also their algebraic properties.\\
For example,
\begin{definition}
    Let $r$ be a real number and $P(x)$ a nonconstant polynomial.\\
    If $r$ is a root of $P(x)$, then it is said to be with multiplicity $m$ if and only if
    $$P(x)=(x-r)^mQ(x)$$
    where $Q(r)\neq 0$ (that is, $r$ is not a root of $Q$).\\
    Otherwise, we define the multiplicity of $r$ in $P$ to be $0$. 
\end{definition}
\begin{remark}
    It is obvious that the multiplicity is well-defined. It is, indeed, the minimal $m$ such that $(x-r)^m$ divides $P$.
\end{remark}
\begin{theorem}
    Let $P(x)$ be a nonconstant polynomial and $r$ a root of it with multiplicity $n$. Then the multiplicity of $r$ in $P^\prime(x)$ is $n-1$.
\end{theorem}
\begin{proof}
    Suppose that $P(x)=(x-r)^nQ(x)$, then
    \begin{align*}
        P^\prime(x)=&n(x-r)^{n-1}Q(x)+(x-r)^nQ^\prime(x)\\
        =&(x-r)^{n-1}(nQ(x)+(x-r)Q^\prime(x))
    \end{align*}
    Now, since $Q(r)\neq 0$, $r$ is not a root of $nQ(x)+(x-r)Q^\prime(x)$ (simply plug it in). This means that $r$ has multiplicity $n-1$ in $P^\prime(x)$.
\end{proof}
There are many other interesting algebraic or even number theorectical results relevant to the derivative of a polynomial.
The Hensel's lemma, for example, might be one of the most fundamental results in building the $p$-adic number theory.

    %TODO
    %\section{The Chain Rule and Derivatives of Rational Powers}
    %\section{Derivatives of Exponentiation, Logarithm and Irrational Powers}
    %\section{Derivatives of trigonometric functions}
    %\section{Implicit Differentiation}
\end{document}
