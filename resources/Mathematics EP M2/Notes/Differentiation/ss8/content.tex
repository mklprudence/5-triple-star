It is really not a very hard thing to believe, but I am not sure why people are buffled by it in their first sight of the idea.
Therefore, I am going to say it here in case some people are confused by it:
If two functions are identical, so are their derivatives.
\begin{theorem}
    If $f=g$, then $f^\prime=g^\prime$.
\end{theorem}
\begin{proof}
    Really? You need me to prove it?
\end{proof}
If it is really so trivial, why do we state it?
One, somewhat obvious, reason, is that not all curves can be written in the form $y=f(x)$.
Some curves simply cannot be described by a function.
The circle being a noticable example.
In this case, we cannot directly use first principle, our rules of differentiation, etc., but we must harness other tricks to obtain the formula of $\mathrm dy/\mathrm dx$ in terms of $x$ and $y$.
\footnote{This is important: Unlike the case of explicit curves, if we want to locate a point on an implicit curve, we almost surely need both the $x$ and $y$ coordinate of it.}
This is how implicit differentiation comes in sight:
We do want to know the slope of tangent to some point on a curve even if it cannot be described by a function.
\begin{example}
    The unit circle is the implicit curve
    $$x^2+y^2=1$$
    If we want to find its slope at a point $(x,y)$ on it, we can do this:
    \begin{align*}
        x^2+y^2=&1\\
        2x+2y\frac{\mathrm dy}{\mathrm dx}=&0\\
        \frac{\mathrm dy}{\mathrm dx}=&-\frac{x}{y}
    \end{align*}
    So, for example, if we want the slope of tangent at $(0.6,0.8)$ (please verify that this point is on the curve), we plug it in and find that it is $-0.75$.
    You may wish to verify that this is true.
\end{example}
Another way of using implicit differentiation is to evaluate derivatives of some bazzire functions.
\begin{example}
    We have looked into the derivative of any real power, but how about the power of itself?
    If $f(x)=x^x$, how do we find $x^x$?
    One
    \footnote{There is another way. See exercises.}
    way is to use implicit differentiation.
    \begin{align*}
        y=&x^x\\
        \ln y=&x\ln x\\
        \frac{1}{y}\frac{\mathrm dy}{\mathrm dx}=&\ln x+1\\
        \frac{\mathrm dy}{\mathrm dx}=&y(\ln x+1)\\
        =&x^x\ln x+x^x
    \end{align*}
\end{example}
Please look into the above example thoroughly and get familiar with this general idea:
We try to simplify the aimed function $f$ by another (simple) function $g$, which is the natual logarithm in this case.
We want to do it in such a way that we know how to differentiate both $g(f(x))$ and $g(x)$.
Assuming that we can, indeed, find such a function $g$, then we do the above procedure over again.
\begin{align*}
    y=&f(x)\\
    g(y)=&g(f(x))\\
    g^\prime(y)\frac{\mathrm dy}{\mathrm dx}=&(g\circ f)^\prime(x)\\
    \frac{\mathrm dy}{\mathrm dx}=&\frac{(g\circ f)^\prime(x)}{g^\prime(y)}\\
    =&\frac{(g\circ f)^\prime(x)}{g^\prime(f(x))}
\end{align*}
Of course, no one will be asking you to memorize the last formula, but you must get a gist of the idea.
Now here is another example.
\begin{example}
    We take $f(x)=\arcsin(x)=\sin^{-1}(x)$ where $f:[0,1]\to[0,\pi/2]$.
    Following the above idea, we (secretly) take $g(x)=\sin(x)$, and this gives us
    \begin{align*}
        y=&f(x)=\arcsin(x)\\
        \sin y=&x\\
        \cos y\frac{\mathrm dy}{\mathrm dx}=&1\\
        \frac{\mathrm dy}{\mathrm dx}=&\frac{1}{\cos(\arcsin x)}\\
        =&\frac{1}{\sqrt{1-x^2}}
    \end{align*}
    The last equality is derived from the identity $\sin^2+\cos^2=1$.
\end{example}
