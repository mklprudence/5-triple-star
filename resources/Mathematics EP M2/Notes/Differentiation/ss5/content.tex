We begin the investigation of the chain rule by some elementary observations.
There was an exercise in a previous section asking for the derivative of the reciprocal of a function from first principle.
If you attempted that one,
\footnote{If you did not, feel free to deduce this from the quotient rule.}
you would remember that we have
$$\frac{\mathrm d}{\mathrm dx}\frac{1}{f(x)}=-\frac{f^\prime(x)}{f(x)^2}$$
Now, there is something strange here: if we substitute $y=f(x)$, the formula is essentially
$$-\frac{1}{y^2}\frac{\mathrm dy}{\mathrm dx}$$
Let the derivative of $y$ alone first, the rest is not something that we see every day. The only recent case where we get acquainted with something looks like $-1/y^2$ is when evaluating
$$\frac{\mathrm d}{\mathrm dy}\frac{1}{y}=-\frac{1}{y^2}$$
But, hey, $1/y$ is exactly $1/f(x)$, the function that we intended to differentiate at first!
There is something fishy here, is it not?\\
If we write $g=1/f$, then we arrive at
$$\frac{\mathrm dg}{\mathrm dx}=\frac{\mathrm dg}{\mathrm dy}\frac{\mathrm dy}{\mathrm dx}$$
This is a nice formula indeed. The ``$\mathrm dy$''s ``cancelled out'' each other and greatly simplifies our evaluation.
Clearly, we cannot \textit{really} cancel out the $\mathrm d$-stuff, because they are not real things.
\footnote{Well, not yet. In some contexts, we have similar things like that, called the \textit{differential forms}, but the theory of that would not be relevant to you if you have not gone through a minimum of two years' study in Maths Major.}
However, that does not mean that we cannot prove it.\\
Now suppose that $g(x)=f(s(x))$, where $f$ and $s$ are both differentiable.
Then, from first principle.
\begin{align*}
    &\lim_{h\to0}\frac{g(x+h)-g(x)}{h}\\
    =&\lim_{h\to0}\frac{f(s(x+h))-f(s(x))}{h}\\
    =&\lim_{h\to0}\frac{f(s(x+h))-f(s(x))}{s(x+h)-s(x)}\frac{s(x+h)-s(x)}{h}
\end{align*}
Since $s$ is differentiable, when $h\to0$, $s(x+h)-s(x)\to0$ (otherwise its derivative would not exist). Hence
$$\lim_{h\to0}\frac{f(s(x+h))-f(s(x))}{s(x+h)-s(x)}=f^\prime(s(x))$$
So
$$\lim_{h\to0}\frac{g(x+h)-g(x)}{h}=f^\prime(s(x))s^\prime(x)$$
This means that $g$ is differentiable and that its derivative is $f^\prime(s(x))s^\prime(x)$.
One may check that this exactly what we have claimed.
Conventionally, we denote $g$ by $f\circ s$
\begin{theorem}
    If $f$ and $g$ are both differentiable, and the domain of $f$ contains the image of $g$ (just to ensure that $f(g(x))$ is always well-defined), then $f\circ g$ is differentible and its derivative is
    $$f^\prime(g(x))g^\prime(x)$$
    or
    $$\frac{\mathrm d(f\circ g)}{\mathrm dx}=\frac{\mathrm d(f\circ g)}{\mathrm dg}\frac{\mathrm dg}{\mathrm dx}$$
\end{theorem}
To avoid some common confusion, the author recommend the first form of the chain rule for all candidates.\\
We now can do some interesting stuff with the chain rule.
A common application of it would be
\begin{example}
    Let $y=e^{x^2+2x}$. By the chain rule (setting $f(x)=e^x$ and $g(x)=x^2+2x$), we have
    $$\frac{\mathrm dy}{\mathrm dx}=f^\prime(g(x))g^\prime(x)=e^{x^2+2x}(2x+2)$$
\end{example}
But not just that.
Using the chain rule, we can also find the derivative of any rational power.
\begin{theorem}
    Let $f(x)=x^r$ where $r=p/q$ for some integers $p,q$, then $f^\prime(x)=rx^{r-1}$.
\end{theorem}
\begin{proof}
    Let $h(x)=x^q$.
    So
    $$px^{p-1}=(h\circ f)^\prime=h^\prime(f(x))f^\prime(x)=qx^{r(q-1)}f^\prime(x)$$
    Rearrange to give
    $$f^\prime(x)=\frac{p}{q}x^{p-1-(p/q)(q-1)}=\frac{p}{q}x^{p/q-1}=rx^{r-1}$$
    As desired.
\end{proof}
Note that in the above arguement we implicitly assumed the differentiability of $f$, which, though obvious, would make a nice exercise.
A warm reminder is that you are making fool of yourself
\footnote{It works, but if you do that, why bother the nice proof above?}
if you directly attempt the first principle on $f$.\\
Now, we have extended our theorem to all rational powers, but how about irrational powers?
If you possess the quality of mathematical rigour, you would have realized that we have not given a definition of irrational powers yet.
Intuitively, for an irrational number $\mathfrak{I}$, if we find a sequence of rational numbers $\mathfrak{r}_n$ which converges to $\mathfrak{I}$, then we may with ton define
$$x^{\mathfrak{I}}=\lim_{n\to\infty}x^{\mathfrak{r}_n}$$
But there are two problems here:\\
Firstly, does the limit exist?
This can be easily answered: yes.
One may attempt that using $\epsilon-\delta$ or get an intuition that if rational numbers $p,q$ are really close, so should be $x^p$ and $x^q$, so the sequence $a_n=x^{\mathfrak{r}_n}$ intuitively converges.\\
Secondly, how do we know that the value of $x^\mathfrak{I}$ is independent of the choice of sequence $\mathfrak{r}_n$?
There can be many totally different sequence that converges to the same value.
How do we know that their difference does not affect the limit?\\
Even if we can answer the second question as well, how could we differentiate it given such conditions?
These will be answered in the next section, along with the rigourous definition of irrational powers. 