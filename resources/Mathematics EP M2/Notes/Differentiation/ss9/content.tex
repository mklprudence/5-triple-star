The derivative of a function is a function.
You already know that.
So it comes right in mind this question:
What happens if we differentiate a function over and over again?
\begin{definition}
    Let $f$ be a function, we define the $n^{th}$ derivative of it inductively as follows:\\
    (1) $f^\prime$ is the first derivative of $f$.
    \footnote{Or sometimes, $f$ is the $0^{th}$ derivative of $f$ itself.}
    \\
    (2) The derivative of the $n^{th}$ derivative of $f$ is the $(n+1)^{th}$ derivative of $f$.
\end{definition}
Now, what are the meanings of higher derivatives?
As derivatives identify the linear approximation of a function, if we take the argument of that function moving along the real line, it essentially gives the rate of change of that function.
So, state it in a clumsy way, the second derivative is the rate of change of the rate of change of the function, and the $n^{th}$ derivative is
$$\text{(the rate of change of )}^n\text{the function}$$
If that is not intuitive enough, recall what you have learnt from your physics lesson.
You should have noted that if we let $s(t)$ to be the displacement function (on a straight line, if you are not familiar with higer dimensions), then $s^\prime$ is the velocity.
Now, the second derivative of $s$, $s^{\prime\prime}$ is just the acceleration.\\
There is not very much additional skill to know to calculate the higher derivatives.
Just rolling the process again and again is fine.
But it is hugely important to know the concept.