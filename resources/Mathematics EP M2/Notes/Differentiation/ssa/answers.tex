\subsection{Section 1}
1. Consider the bahaviour of the function $f(x)=|x|$ at $x=0$.

\subsection{Section 2}
1.
\begin{align*}
    \lim_{h\to0}\frac{\frac{1}{x+h}-\frac{1}{x}}{h}
    =&\lim_{h\to0}\frac{-h}{hx(x+h)}\\
    =&\lim_{h\to0}-\frac{1}{x(x+h)}\\
    =&-\frac{1}{x^2}
\end{align*}
Therefore,
$$f^\prime(x)=\lim_{h\to0}\frac{f(x+h)-f(x)}{h}=-\frac{1}{x^2}$$
2.
\begin{align*}
    \lim_{h\to0}\frac{\sqrt{x+h}-\sqrt{x}}{h}
    =&\lim_{h\to0}\frac{h}{h\sqrt{x}+\sqrt{x+h}}\\
    =&\lim_{h\to0}\frac{1}{\sqrt{x}+\sqrt{x+h}}\\
    =&\frac{1}{2\sqrt{x}}
\end{align*}
Therefore,
$$f^\prime(x)=\lim_{h\to0}\frac{f(x+h)-f(x)}{h}=\frac{1}{2\sqrt{x}}$$
3.
\begin{align*}
    \lim_{h\to0}\frac{e^{ax+ah}-e^{ax}}{h}
    =&\lim_{h\to0}ae^{ax}\frac{e^{ah}-1}{ah}\\
    =&(ae^{ax})(1)\\
    =&ae^{ax}
\end{align*}
Therefore,
$$f^\prime(x)=\lim_{h\to0}\frac{f(x+h)-f(x)}{h}=ae^{ax}$$
4. (a)
\begin{align*}
    \lim_{h\to0}\frac{\sin(x+h)-\sin(x)}{h}
    =&\lim_{h\to0}\frac{2\cos(x+h/2)\sin(h/2)}{h}\\
    =&\lim_{h\to0}\cos(x+h/2)\frac{\sin(h/2)}{h/2}\\
    =&\cos(x)
\end{align*}
Therefore,
$$f^\prime(x)=\lim_{h\to0}\frac{f(x+h)-f(x)}{h}=\cos(x)$$
(b)
\begin{align*}
    \lim_{h\to0}\frac{\cos(x+h)-\cos(x)}{h}
    =&-\lim_{h\to0}\frac{\sin(x-\pi/2+h)-\sin(x-\pi/2)}{h}\\
    =&\left.\frac{\mathrm d\sin(z)}{\mathrm dz}\right|_{z=x-\pi/2}\\
    =&-\cos(x-\pi/2)\\
    =&-\sin(x)
\end{align*}
Therefore,
$$f^\prime(x)=\lim_{h\to0}\frac{f(x+h)-f(x)}{h}=-\sin(x)$$
5.
\begin{align*}
    \lim_{h\to0}\frac{g(x+h)-g(x)}{h}
    =&\lim_{h\to0}\frac{cf(x+h)-cf(x)}{h}\\
    =&c\lim_{h\to0}\frac{f(x+h)-f(x)}{h}\\
    =&cf^\prime(x)
\end{align*}
Therefore,
$$f^\prime(x)=\lim_{h\to0}\frac{g(x+h)-g(x)}{h}=cf^\prime(x)$$
\subsection{Section 3}
1. 
$$\frac{\mathrm d}{\mathrm dx}\frac{\sqrt{x}}{e^x}=\frac{\frac{1}{2\sqrt{x}}e^x-\sqrt{x}e^x}{e^{2x}}=\frac{1-2x}{2\sqrt{x}e^x}$$
2. (a)
$$\frac{\mathrm d}{\mathrm dx}x^3=(1)x^2+x(2x)=3x^2$$
$$\frac{\mathrm d}{\mathrm dx}x^4=(1)x^3+x(3x^2)=4x^3$$
(b) Let $P(n)$ be the proposition. When $n=1$, $\mathrm{L.H.S.}=1=\mathrm{R.H.S.}$, so $P(1)$ is true. Now assume that $P(k)$ is true for some positive integer $k$, then when $n=k+1$,
\begin{align*}
    \mathrm{L.H.S.}
    =&\frac{\mathrm d}{\mathrm dx}x^{k+1}\\
    =&\left(\frac{\mathrm d}{\mathrm dx}x\right)x^k+x\left(\frac{\mathrm d}{\mathrm dx}x^k\right)\\
    =&x^k+x(kx^{k-1})\\
    =&(k+1)x^k=\mathrm{R.H.S.}
\end{align*}
So $P(k+1)$ is true. By the principle of mathematical induction, $P(n)$ is true for all positive integer $n$.\\
3. (a)
\begin{align*}
    \lim_{h\to0}\frac{\frac{1}{f(x+h)}-\frac{1}{f(x)}}{h}
    =&-\lim_{h\to0}\frac{1}{f(x)f(x+h)}\frac{f(x+h)-f(x)}{h}\\
    =&-\frac{f^\prime(x)}{f(x)^2}
\end{align*}
(b)
\begin{align*}
    \frac{\mathrm d}{\mathrm dx}\frac{f(x)}{g(x)}
    =&\frac{\mathrm d}{\mathrm dx}f(x)\frac{1}{g(x)}\\
    =&\left(\frac{\mathrm d}{\mathrm dx}f(x)\right)\frac{1}{g(x)}+f(x)\left(\frac{\mathrm d}{\mathrm dx}\frac{1}{g(x)}\right)\\
    =&\frac{f^\prime(x)}{g(x)}-\frac{f(x)g^\prime(x)}{g(x)^2}\\
    =&\frac{f^\prime(x)g(x)-f(x)g^\prime(x)}{g(x)^2}
\end{align*}

\subsection{Section 4}
1. $x^9e^x+9x^8e^x$.\\
2. $5x^4-1$.

\subsection{Section 5}
1. $\frac{e^{\sqrt{x}}}{2\sqrt{x}}$

\subsection{Section 6}
1. $ex^{e-1}-e^x$

\subsection{Section 7}
1. (3)
$$\frac{\mathrm d}{\mathrm dx}\tan(x)=\frac{\mathrm d}{\mathrm dx}\frac{\sin(x)}{\cos(x)}=\frac{\sin^2(x)+\cos^2(x)}{\cos^2(x)}=\frac{1}{\cos^2(x)}=\sec^2(x)$$
(4)
$$\frac{\mathrm d}{\mathrm dx}\sec(x)=\frac{\mathrm d}{\mathrm dx}\frac{1}{\cos(x)}=-\frac{1}{\cos^2(x)}(-\sin(x))=\sec(x)\tan(x)$$
(5)
$$\frac{\mathrm d}{\mathrm dx}\cot(x)=\frac{\mathrm d}{\mathrm dx}\frac{\cos(x)}{\sin(x)}=\frac{-\sin^2(x)-\cos^2(x)}{\sin^2(x)}=-\frac{1}{\sin^2(x)}=-\csc^2(x)$$
(6)
$$\frac{\mathrm d}{\mathrm dx}\csc(x)=\frac{\mathrm d}{\mathrm dx}\frac{1}{\sin(x)}=-\frac{1}{\sin^2(x)}(\cos(x))=-\csc(x)\cot(x)$$
2. (3)
\begin{align*}
    \lim_{h\to0}\frac{\tan(x+h)-\tan(x)}{h}
    =&\lim_{h\to0}\frac{\sin(x+h)\cos(x)-\cos(x+h)\sin(x)}{h\cos(x)\cos(x+h)}\\
    =&\lim_{h\to0}\frac{\sin(h)}{h}\frac{1}{\cos(x)\cos(x+h)}\\
    =&\sec^2(x)
\end{align*}
Therefore,
$$\frac{\mathrm d\tan(x)}{\mathrm dx}=\sec^2(x)$$
(4)
\begin{align*}
    \lim_{h\to0}\frac{\sec(x+h)-\sec(x)}{h}
    =&\lim_{h\to0}\frac{\cos(x)-\cos(x+h)}{h\cos(x)\cos(x+h)}\\
    =&\lim_{h\to0}\frac{-2\sin(x+h/2)\sin(-h/2)}{h\cos(x)\cos(x+h)}\\
    =&\lim_{h\to0}\frac{\sin(x+h/2)}{\cos(x)\cos(x+h)}\frac{\sin(h/2)}{h/2}\\
    =&\frac{\sin(x)}{\cos^2(x)}\\
    =&\sec(x)\tan(x)
\end{align*}
Therefore,
$$\frac{\mathrm d\sec(x)}{\mathrm dx}=\sec(x)\tan(x)$$
(5)
\begin{align*}
    \lim_{h\to0}\frac{\cot(x+h)-\cot(x)}{h}
    =&\lim_{h\to0}\frac{\sin(x)\cos(x+h)-\cos(x)\sin(x+h)}{h\sin(x)\sin(x+h)}\\
    =&\lim_{h\to0}-\frac{\sin(h)}{h}\frac{1}{\sin(x)\sin(x+h)}\\
    =&-\csc^2(x)
\end{align*}
Therefore,
$$\frac{\mathrm d\cot(x)}{\mathrm dx}=-\csc^2(x)$$
(6)
\begin{align*}
    \lim_{h\to0}\frac{\csc(x+h)-\csc(x)}{h}
    =&\lim_{h\to0}\frac{\sin(x)-\sin(x+h)}{h\sin(x)\sin(x+h)}\\
    =&\lim_{h\to0}\frac{2\cos(x+h/2)\sin(-h/2)}{h\sin(x)\sin(x+h)}\\
    =&\lim_{h\to0}-\frac{\cos(x+h/2)}{\sin(x)\sin(x+h)}\frac{\sin(h/2)}{h/2}\\
    =&-\frac{\cos(x)}{\sin^2(x)}\\
    =&-\csc(x)\cot(x)
\end{align*}
Therefore,
$$\frac{\mathrm d\csc(x)}{\mathrm dx}=-\csc(x)\cot(x)$$
3.
$$f^\prime(x)=\cos(x^2\cos(e^x))(2x\cos(e^x)-x^2\sin(e^x)e^x)$$

\subsection{Section 8}
1. We have $x^x=e^{x\ln x}$, so
$$\frac{\mathrm dx^x}{\mathrm dx}=\frac{\mathrm de^{x\ln x}}{\mathrm dx}=e^{x\ln x}(\ln x+x\frac{1}{x})=x^x(\ln x+1)$$
2.
\begin{align*}
    ye^y=&\ln x\\
    (ye^y+e^y)\frac{\mathrm dy}{\mathrm dx}=&\frac{1}{x}\\
    \frac{\mathrm dy}{\mathrm dx}=&\frac{1}{xye^y+xe^y}
\end{align*}
So when $x=1,y=0$, we have
$$\left.\frac{\mathrm dy}{\mathrm dx}\right|_{(1,0)}=1$$

\subsection{Section 9}
1. (a) When $f(x)=\sin(x)$,
$$\mathrm{L.H.S.}=f^{\prime\prime}=\frac{\mathrm d}{\mathrm dx}\cos(x)=-\sin(x)=-f=\mathrm{R.H.S.}$$
When $f(x)=\cos(x)$,
$$\mathrm{L.H.S.}=f^{\prime\prime}=\frac{\mathrm d}{\mathrm dx}-\sin(x)=-\cos(x)=-f=\mathrm{R.H.S.}$$
(b) When $f(x)=A\cos(x)+B\sin(x)$,
\begin{align*}
    \mathrm{L.H.S.}
    =&f^{\prime\prime}\\
    =&\frac{\mathrm d}{\mathrm dx}(-A\sin(x)+B\cos(x))\\
    =&-A\cos(x)-B\sin(x)\\
    =&-f=\mathrm{R.H.S.}
\end{align*}
