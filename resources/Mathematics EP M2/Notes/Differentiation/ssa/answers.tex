\subsection{Section 1}
1. Consider the bahaviour of the function $f(x)=|x|$ at $x=0$.

\subsection{Section 2}
1.
\begin{align*}
    \lim_{h\to0}\frac{\frac{1}{x+h}-\frac{1}{x}}{h}
    =&\lim_{h\to0}\frac{-h}{hx(x+h)}\\
    =&\lim_{h\to0}-\frac{1}{x(x+h)}\\
    =&-\frac{1}{x^2}
\end{align*}
Therefore,
$$f^\prime(x)=\lim_{h\to0}\frac{f(x+h)-f(x)}{h}=-\frac{1}{x^2}$$
2.
\begin{align*}
    \lim_{h\to0}\frac{\sqrt{x+h}-\sqrt{x}}{h}
    =&\lim_{h\to0}\frac{h}{h\sqrt{x}+\sqrt{x+h}}\\
    =&\lim_{h\to0}\frac{1}{\sqrt{x}+\sqrt{x+h}}\\
    =&\frac{1}{2\sqrt{x}}
\end{align*}
Therefore,
$$f^\prime(x)=\lim_{h\to0}\frac{f(x+h)-f(x)}{h}=\frac{1}{2\sqrt{x}}$$
3.
\begin{align*}
    \lim_{h\to0}\frac{e^{ax+ah}-e^{ax}}{h}
    =&\lim_{h\to0}ae^{ax}\frac{e^{ah}-1}{ah}\\
    =&(ae^{ax})(1)\\
    =&ae^{ax}
\end{align*}
Therefore,
$$f^\prime(x)=\lim_{h\to0}\frac{f(x+h)-f(x)}{h}=ae^{ax}$$
4. (a)
\begin{align*}
    \lim_{h\to0}\frac{\sin(x+h)-\sin(x)}{h}
    =&\lim_{h\to0}\frac{2\cos(x+h/2)\sin(h/2)}{h}\\
    =&\lim_{h\to0}\cos(x+h/2)\frac{\sin(h/2)}{h/2}\\
    =&\cos(x)
\end{align*}
Therefore,
$$f^\prime(x)=\lim_{h\to0}\frac{f(x+h)-f(x)}{h}=\cos(x)$$
(b)
\begin{align*}
    \lim_{h\to0}\frac{\cos(x+h)-\cos(x)}{h}
    =&-\lim_{h\to0}\frac{\sin(x-\pi/2+h)-\sin(x-\pi/2)}{h}\\
    =&\left.\frac{\mathrm d\sin(z)}{\mathrm dz}\right|_{z=x-\pi/2}\\
    =&-\cos(x-\pi/2)\\
    =&-\sin(x)
\end{align*}
Therefore,
$$f^\prime(x)=\lim_{h\to0}\frac{f(x+h)-f(x)}{h}=-\sin(x)$$
5.
\begin{align*}
    \lim_{h\to0}\frac{g(x+h)-g(x)}{h}
    =&\lim_{h\to0}\frac{cf(x+h)-cf(x)}{h}\\
    =&c\lim_{h\to0}\frac{f(x+h)-f(x)}{h}\\
    =&cf^\prime(x)
\end{align*}
Therefore,
$$f^\prime(x)=\lim_{h\to0}\frac{g(x+h)-g(x)}{h}=cf^\prime(x)$$
\subsection{Section 3}
1. 
$$\frac{\mathrm d}{\mathrm dx}\frac{\sqrt{x}}{e^x}=\frac{\frac{1}{2\sqrt{x}}e^x-\sqrt{x}e^x}{e^{2x}}=\frac{1-2x}{2\sqrt{x}e^x}$$
2. (a)
$$\frac{\mathrm d}{\mathrm dx}x^3=(1)x^2+x(2x)=3x^2$$
$$\frac{\mathrm d}{\mathrm dx}x^4=(1)x^3+x(3x^2)=4x^3$$
(b) Let $P(n)$ be the proposition. When $n=1$, $\mathrm{L.H.S.}=1=\mathrm{R.H.S.}$, so $P(1)$ is true. Now assume that $P(k)$ is true for some positive integer $k$, then when $n=k+1$,
\begin{align*}
    \mathrm{L.H.S.}
    =&\frac{\mathrm d}{\mathrm dx}x^{k+1}\\
    =&\left(\frac{\mathrm d}{\mathrm dx}x\right)x^k+x\left(\frac{\mathrm d}{\mathrm dx}x^k\right)\\
    =&x^k+x(kx^{k-1})\\
    =&(k+1)x^k=\mathrm{R.H.S.}
\end{align*}
So $P(k+1)$ is true. By the principle of mathematical induction, $P(n)$ is true for all positive integer $n$.\\
3. (a)
\begin{align*}
    \lim_{h\to0}\frac{\frac{1}{f(x+h)}-\frac{1}{f(x)}}{h}
    =&-\lim_{h\to0}\frac{1}{f(x)f(x+h)}\frac{f(x+h)-f(x)}{h}\\
    =&-\frac{f^\prime(x)}{f(x)^2}
\end{align*}
(b)
\begin{align*}
    \frac{\mathrm d}{\mathrm dx}\frac{f(x)}{g(x)}
    =&\frac{\mathrm d}{\mathrm dx}f(x)\frac{1}{g(x)}\\
    =&\left(\frac{\mathrm d}{\mathrm dx}f(x)\right)\frac{1}{g(x)}+f(x)\left(\frac{\mathrm d}{\mathrm dx}\frac{1}{g(x)}\right)\\
    =&\frac{f^\prime(x)}{g(x)}-\frac{f(x)g^\prime(x)}{g(x)^2}\\
    =&\frac{f^\prime(x)g(x)-f(x)g^\prime(x)}{g(x)^2}
\end{align*}

\subsection{Section 4}
\subsection{Section 5}
\subsection{Section 6}
%\subsection{Derivatives of trigonometric functions}
%\subsection{Implicit Differentiation}
%\subsection{Higher derivatives}