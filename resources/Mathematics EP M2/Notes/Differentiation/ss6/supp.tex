We now know how to differentiate any real powers, and they all have the same form.
An interesting question, then, is if we can extend our journey to the field of complex numbers, $\mathbb C$.\\
The answer, however, is not promising.
We can't.
Why?
Well, firstly the logarithmatic function might not be in a form which you will like it.
Euler's theorem tells us that
$$z=|z|e^{i\arg{z}}$$
So in order to let the logarithmatic function makes sence, we must have
$$\ln{z}=\ln|z|+i\arg{z}$$
Can't see any problem here?
Take a look at the latter part of the expression: $\arg{z}$.
$\arg{z}$ is defined in a way that it only spans over a range of $2\pi$, otherwise it will be ill-defined.
But it simply does not work. Choose two complex numbers $z,w$ such that the sum of their arguments is not in your chosen interval of length $2\pi$.
We can obviouly choose them no matter which interval we chose for the $\arg$ function.
But then
$$\ln|w|+\ln|z|+i(\arg{w}+\arg{z})=\ln{w}+\ln{z}=\ln{wz}=\ln|w|+\ln|z|+i\arg{wz}$$
But this is false: $\arg{w}+\arg{z}$ is simply not in the image of $\arg$!
So we cannot let the logarithmatic function have the same crucial properties that we will expect them to inherit from the real number case.
A way this goes ill-defined is the evaluation of $i^i$.\\
But is there a way to well-define these powers?
The original way does not work, but is there any other way that works?
In fact, there is no way that well-defines complex exponentiation in the way we wish, which you may want to find out why yourself.
