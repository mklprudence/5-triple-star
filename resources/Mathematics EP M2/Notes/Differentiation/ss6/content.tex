We already know that the derivative of the exponential function is itself.
That is
$$\frac{\mathrm de^x}{\mathrm dx}=e^x$$
It is natual for us to enquire the case for the opposite: the logarithm.\\
There is nothing very polished here, just the usual first principle.
\begin{align*}
    \lim_{h\to0}\frac{\ln(x+h)-\ln(x)}{h}
    =&\lim_{h\to0}\ln\left(\left(1+\frac{1}{x}h\right)^{1/h}\right)\\
    =&\ln\left(\lim_{h\to0}\left(1+\frac{1}{x}h\right)^{1/h}\right)\\
    =&\ln(e^{1/x})\\
    =&\frac{1}{x}
\end{align*}
Therefore the logarithm function is differentiable and its derivative is $1/x$.\\
Using the logarithmatic function, we could finally give a more genuine definition of irrational powers (or, in general, real powers).
Observe first that if $r$ is any rational number then
$$x^r=e^{r\ln{x}}$$
What is the advantage of formulating it like this?
Well, one may notice that $r$ is now the coefficient of $\ln{x}$ on the exponent, so the expression makes sense if we substitute $r$ by simply any real number.
The core reason why this can happen is because we have defined the exponentiation of the natural base in a way general enough to include all real numbers.
So what we have tried to do is to let the exponentiation of any real number inherit this nice property of our beloved $e$.\\
Of course, an expression that makes sense (well-defined) cannot let us to accept it as a notion.
There are more important properties to be checked.
One may attempt oneself that if we define a real power by the mannar
$$x^y=e^{y\ln{x}}$$
then the set of rules of exponentiation hold still.
A more important property stated at the end of the last section can also be checked.
For any sequence $\mathfrak r_n$ converging to a real number $\mathfrak I$, then the continuity of $e^x$ and the multiplication operation guarantee that
$$\lim_{n\to\infty}x^{\mathfrak r_n}=\lim_{n\to\infty}e^{\mathfrak r_n\ln{x}}=e^{\ln{x}\lim_{n\to\infty}\mathfrak r_n}=e^{\mathfrak I\ln{x}}=x^{\mathfrak I}$$
So this way of defining the general power gives it properties that perfectly resemble what we expect it to have.\\
Now, how about its derivative?
\begin{theorem}
    Let $y$ be any real number, we have
    $$\frac{\mathrm dx^y}{\mathrm dx}=yx^{y-1}$$
\end{theorem}
\begin{proof}
    The chain rule works.
    $$\frac{\mathrm dx^y}{\mathrm dx}=e^{y\ln x}\frac{y}{x}=yx^{y-1}$$
    As desired.
\end{proof}
Our chasing of the derivaive of powers finally ends.
But this is hardly the end of story.
Many other sorts of functions exist.
Trigonometric functions, for example, deserves attention.
This introduces us the next section.