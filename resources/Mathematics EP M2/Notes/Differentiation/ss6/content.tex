We already know that the derivative of the exponential function is itself.
That is
$$\frac{\mathrm de^x}{\mathrm dx}=e^x$$
It is natual for us to enquire the case for the opposite: the logarithm.\\
There is nothing very polished here, just the usual first principle.
\begin{align*}
    \lim_{h\to0}\frac{\ln(x+h)-\ln(x)}{h}
    =&\lim_{h\to0}\ln\left(\left(1+\frac{1}{x}h\right)^{1/h}\right)\\
    =&\ln\left(\lim_{h\to0}\left(1+\frac{1}{x}h\right)^{1/h}\right)\\
    =&\ln(e^{1/x})\\
    =&\frac{1}{x}
\end{align*}
Therefore the logarithm function is differentiable and its derivative is $1/x$.