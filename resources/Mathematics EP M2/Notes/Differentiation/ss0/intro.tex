What is differentiation?
Broadly speaking, differentiation is the study of the \textit{linear approximation} of a function, or, more appetizingly, the tendency of the trend that some function $f(t)$ will take when $t$ increases.\\
The studies of approximation, trends and optima are long-standing problems of many aspects of development like engineering.
The first sight of differentiation can even be found in ancient Greek, Japan and China.\\
This note is divided into 3 parts.
The first two sections are purely about the intuition behind the definition of derivative (the first principle), for those who think that this definition is weird.
From the third section onwards, we focus on the techniques and laws that can help us differentiate a function without having to cast the first principle.
This includes product rule, quotient rule, chain rule, etc..
Also, we will go through the derivatives of some special functions, like logarithm and trigonometry.
The last few section are about some notions relating to derivatives, which you ought to know for one reason or another.\\
It is recommended that you follow the chapters in their given order.
\ifcompilesupp
However, you may skip the supplementary information part at the end of some sections.
\fi
It will be best if you finish all the exercises, which is not that many.
Sometimes an exercise does not require knowledge from the section which it is in.
They are there for sake of a better understanding or alternate proofs.
Those exercises are marked with an asterisk (*).
\\
Throughout this note, unless otherwise specified, all the functions take value in $\mathbb R$ and their domains are open sets (unions of some open intervals) in $\mathbb R$.