What is differentiation? Broadly speaking, differentiation is the study of the so-called \textit{linear approximation} of a function, or, more appetizingly, the tendency of the trend that some function $f(t)$ will take when $t$ increases.\\
The studies of approximation, trends and optima are long-standing problems of many aspects of development like engineering.
The first sight of differentiation can even be found in ancient Greek, Japan and China.\\
This note is divided into 3 parts.
The first two sections are purely about the intuition behind the definition of derivative (the first principle), for those who think that this definition is weird.
From the third section onwards, we focus on the techniques and the laws that helps us to differentiate a function without having to attempt the first principle everytime.
This includes product rule, quotient rule, chain rule, etc..
The last few sections are concerning the application of differentiation to different problems, which is a heat topic in exams.\\
Sometimes there are some supplementary information at the end of a section. All of them are skippable.\\
Throughout this note, unless otherwise specified, we are interested in functions $I\to\mathbb R$, where $I$ is a union of some open intervals in $\mathbb R$.