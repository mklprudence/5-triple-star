From this section on, we are going to introduce some rules that we can use to differentiate a function more easily.
\footnote{Author of some Calculus testbook prefer the wording: ``Construct more differentiable functions from existing ones''.}
We already know that we can take a multiplicative constant out of the differentiation sign from an exercise in the preceding session, but we can actually do something even more useful.
The first things that we will look at in the forcoming sections will be the derivatives of the sums, products and quotients (or fractions) of differentiable functions.
Now, suppose $f$ and $g$ are differentiable, then
\begin{align*}
    &\lim_{h\to0}\frac{f(x+h)+g(x+h)-(f(x)+g(x))}{h}\\
    =&\lim_{h\to0}(\frac{f(x+h)-f(x)}{h}+\frac{g(x+h)-g(x)}{h})\\
    =&\lim_{h\to0}\frac{f(x+h)-f(x)}{h}+\lim_{h\to0}\frac{g(x+h)-g(x)}{h}\\
    =&f^\prime(x)+g^\prime(x)
\end{align*}
Therefore, the function $f+g$ is differentiable as well, and its derivative is $f^\prime+g^\prime$.
\begin{example}
    Let $f(x)=x^2+3x$. We have already knwon how to differentiate $x^2$ and $3x$ alone respectively from the preceding section, so by what we have already got,
    $$f^\prime(x)=(2x)+(3)=2x+3$$
\end{example}
We can do the same thing for products, of course. We of course start with our beloved first principle:
\begin{align*}
    &\lim_{h\to0}\frac{f(x+h)g(x+h)-f(x)g(x)}{h}\\
    =&\lim_{h\to0}f(x+h)\frac{g(x+h)-g(x)}{h}+\frac{f(x+h)-f(x)}{h}g(x)\\
    =&\lim_{h\to0}f(x+h)\lim_{h\to0}\frac{g(x+h)-g(x)}{h}+\lim_{h\to0}\frac{f(x+h)-f(x)}{h}\lim_{h\to0}g(x)\\
    =&f(x)g^\prime(x)+f^\prime(x)g(x)
\end{align*}
So $f\cdot g$ is differentiable and its derivative is $f\cdot g^\prime+f^\prime\cdot g$.
\begin{example}
    We now try to attempt to differentiate $f(x)=x^2e^x$. We know the derivatives of $x^2$ and $e^x$ are $2x$ and $e^x$ respectively, so
    $$f^\prime(x)=(2x)e^x+x^2(e^x)=2xe^x+x^2e^x$$
\end{example}
After sums and products, we arrive as quotients (fractions), which, however, is not as neat as the sum and product case.
There are essentially many ways of getting the quotient rule.
The simpliest of which is to think of $f/g$ as $f\cdot 1/g$ and use the product rule to proceed.
However, we have not covered the ways to differentiate the reciprocal of a function
\footnote{You may consult the session about the chain rule later.}
yet.
So we will exhibit the first principle solution here.
\begin{align*}
    &\lim_{h\to0}\frac{\frac{f(x+h)}{g(x+h)}-\frac{f(x)}{g(x)}}{h}\\
    =&\lim_{h\to0}\frac{1}{g(x+h)g(x)}\frac{f(x+h)g(x)-f(x)g(x+h)}{h}\\
    =&\lim_{h\to0}\frac{1}{g(x+h)g(x)}
    \left(\frac{f(x+h)-f(x)}{h}g(x)-f(x)\frac{g(x+h)-g(x)}{h}\right)\\
    =&\frac{\lim_{h\to0}\frac{f(x+h)-f(x)}{h}g(x)-f(x)\lim_{h\to0}\frac{g(x+h)-g(x)}{h}}{\lim_{h\to0}g(x+h)g(x)}\\
    =&\frac{f^\prime(x)g(x)-f(x)g^\prime(x)}{g(x)^2}
\end{align*}
The quotient rule might not be as easy as the two preceding rules for memorization, but it is still essential.
Just remember the square in the denominator, minus sign in the nominator and the fact that $f^\prime$ goes before $g^\prime$.