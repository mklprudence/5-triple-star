From this section on, we are going to introduce some rules that we can use to differentiate a function more easily.
The first thing that we will look at is the sums, products and quotients (or fractions).
Now, suppose $f$ and $g$ are differentiable, then
\begin{align*}
    &\lim_{h\to0}\frac{f(x+h)+g(x+h)-(f(x)+g(x))}{h}\\
    =&\lim_{h\to0}(\frac{f(x+h)-f(x)}{h}+\frac{g(x+h)-g(x)}{h})\\
    =&\lim_{h\to0}\frac{f(x+h)-f(x)}{h}+\lim_{h\to0}\frac{g(x+h)-g(x)}{h}\\
    =&f^\prime(x)+g^\prime(x)
\end{align*}
Therefore, the function $f+g$ is differentiable as well, and its derivative is $f^\prime+g^\prime$.