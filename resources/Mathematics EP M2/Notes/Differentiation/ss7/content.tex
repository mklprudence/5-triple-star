You have found that the derivative of $\sin(x)$ is $\cos(x)$ from a previous exercise.
From the very same exercise which you are told so, you should also have proved that the derivative of $\cos(x)$ is $-\sin(x)$.
The argument, not using the first principle or any other rule, is good to learn, so I am not spoiling your fun.
The standard way of differentiating $\cos(x)$ is just sum-to-product all over again.
Since it is not in any exercise (yet), I will demonstrate here:
\begin{align*}
    \lim_{h\to0}\frac{\cos(x+h)-\cos(x)}{h}=&
    \lim_{h\to0}\frac{-2\sin(x+\frac{h}{2})\sin(\frac{h}{2})}{h}\\
    =&\lim_{h\to0}-\sin(x+\frac{h}{2})\frac{\sin(\frac{h}{2})}{\frac{h}{2}}\\
    =&-\sin(x)
\end{align*}
Virtually all trigonometric functions are built from sines and cosines.
So you will be expected to prove the following:
\begin{theorem}
    All trigonometric functions are differentiable in their respective domains.
    Plus,
    \begin{align}
        \frac{\mathrm d\sin(x)}{\mathrm dx}=&\cos(x)\\
        \frac{\mathrm d\cos(x)}{\mathrm dx}=&-\sin(x)\\
        \frac{\mathrm d\tan(x)}{\mathrm dx}=&\sec^2(x)\\
        \frac{\mathrm d\sec(x)}{\mathrm dx}=&\sec(x)\tan(x)\\
        \frac{\mathrm d\cot(x)}{\mathrm dx}=&-\csc^2(x)\\
        \frac{\mathrm d\csc(x)}{\mathrm dx}=&-\csc(x)\cot(x)
    \end{align}
    \label{trigo}
\end{theorem}
From here on we can find the derivative of some rather strange functions by a combination of Theorem \ref{trigo} and the chain rule.
\begin{example}
    Let $f(x)=\tan(\ln(x))$, we have $f^\prime(x)=\sec^2(\ln(x))/x$.
\end{example}
\begin{example}
    Let $f(x)=\sin(\sin(x))\sin(x)$, we have
    $$f^\prime(x)=\sin(\sin(x))\cos(x)+\cos(\sin(x))\cos(x)\sin(x)$$
\end{example}