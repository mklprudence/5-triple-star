If we want to see the behaviour of a function $g(x)$ when $a,b$ are really close, we take the limit
$$\lim_{b\to a}g(x)$$
Similarly, if we want to see the linear approximation of a function around $a$, we can consider 
$$\lim_{b\to a}\frac{f(b)-f(a)}{b-a}$$
or
$$\lim_{h\to 0}\frac{f(a+h)-f(a)}{h}$$
By the substitution $h=b-a$. Of course, this limit may or may not exist, but it does exist almost anywhere in almost all the functions ordinary people
\footnote{This excludes mathematicians, of course.}
will ever encounter.
\begin{definition}
    If the above limit exists, we say that $f$ is \textit{differentiable} at $a$.
    If $f$ is differentiable in a subset $S$ of the domain, the function $f^\prime:S\to\mathbb R$ defined by 
    $$f^\prime(a)=\lim_{h\to 0}\frac{f(a+h)-f(a)}{h}$$
    is called the \textit{derivative} of $f$.
\end{definition}
\begin{remark}
    The author prefer to denote the derivative of $y=f(x)$ as $f^\prime(x)$, but there is often other notations like $\dot{y}$ or $\frac{\mathrm dy}{\mathrm dx}$.
    The difference of notations results from the rather complex history of the development of Calculus.
\end{remark}
We certainly have the freedom to attempt evaluating the limit for some functions we like, and it is highly encouraged that you attempt evaluating this limit for some of your favourite functions.
\begin{example}
    We take $f(x)=x^2$, so
    $$\lim_{h\to0}\frac{f(x+h)-f(x)}{h}=\lim_{h\to0}\frac{2xh+h^2}{h}=\lim_{h\to0}(2x+h)=2x$$
    Therefore,
    $$f^\prime(x)=\lim_{h\to0}\frac{f(x+h)-f(x)}{h}=2x$$
\end{example}
\begin{remark}
    When attempting evaluating a derivative, it is better NOT to write, in the beginning, something like
    $$f^\prime(x)=\lim_{h\to0}\frac{f(x+h)-f(x)}{h}=\cdots$$
    because we DO NOT YET know if the limit really exists. So one should attempt evaluating the limit first before concluding the derivative of the function.
\end{remark}
\begin{example}
    We can also find the derivative of a function at some specific point. Take the function $f(x)=\exp(x)=e^x$, if we want to find its derivative at $x=0$, we do
    $$\lim_{h\to0}\frac{e^{0+h}-e^0}{h}=\lim_{h\to0}\frac{e^h-1}{h}=1$$
    So $f^\prime(0)=1$.
    \footnote{Actually, the derivative of $e^x$ is $e^x$ itself, which you may wish to verify. And a more interesting thing is that the \textit{only} functions whose derivatives are themselves are in the form $c\cdot e^x$ where $c$ is a constant.}
\end{example}
Suppose $f(x)=mx+c$ for some constants $m$ and $c$, then one may verify that $f^\prime$ is exactly the constant function $m$. What this gives us is exactly what we expect to be the geometrical meaning of a derivative: the slope of the tangent of a (differentiable) curve!