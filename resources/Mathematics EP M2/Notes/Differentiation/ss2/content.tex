If we want to see the behaviour of a function $g(x)$ when $a,b$ to be really close, we take the limit
$$\lim_{b\to a}g(x)$$
Similarly, if we want to see the ultimate linear approximation of a function at point $a$, we can consider 
$$\lim_{b\to a}\frac{f(b)-f(a)}{b-a}$$
or
$$\lim_{h\to 0}\frac{f(a+h)-f(a)}{h}$$
By the substitution $h=b-a$. Of course, this limit may or may not exist, but it does exist almost anywhere in almost all the functions you will encounter in daily life.
For example, if $f$ is, actually, linear with slope $m$, one may verify that the expression in the first principle essentially gives $m$.
