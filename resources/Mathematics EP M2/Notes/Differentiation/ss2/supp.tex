There is yet another intepretation of the derivative notion.
If we want to directly conduct a linear approximation of the function near a point $x$, we are actually hoping to find a constant $m$ such that $f(x)$ \textit{behaves like} $mx+c$ for some onstant $c$.
In other words, we want
$$f(x+h)=f(x)+mh+o(h)$$
for some small $h$.
What is $o(h)$? It is what we call an \textit{error term}.
Since we can only \textit{approximate} the function by a linear tangent, there should normally be some difference between this linear function and the original function.
\footnote{N.B. $o(h)$ depends on $h$. Also, for computer scientists, this $o(h)$ is exactly the little-$o$ notation that you use when bounding time complexities}
This difference is the error term that allows us to use $=$ instead of $\approx$.\\
But there should be a restriction on the magnitude of $o(h)$, or $m$ could just be any real number.
Now, since we want the function to resemble that linear function as close as we please when $h$ is small enough, we will want $o(h)$ to go to $0$ faster than $h$ does.
In the language of limit, this means that
$$\lim_{h\to0}\frac{o(h)}{h}=0$$
So, if we rearrange the equation, we will find that
$$\frac{f(x+h)-f(x)}{h}=m+\frac{o(h)}{h}\to m\text{ as }h\to0$$
This gives, again, the first principle. One may check that the existence of such an $m$ is equivalent to the differentiability of $f$ at point $x$.