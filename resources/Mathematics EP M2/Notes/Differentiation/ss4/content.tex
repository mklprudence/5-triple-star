The most beloved family of functions for an algebraist is polynomials.
It is quite right to say that polynomials are indeed the most important set of functions mathematicians study throughout the history of mathematics.
So it seems only sensible that we would want to study their properties by every tool that we het in hand.
In this case, that tool would be differentiation.\\
From the exercise from the previous section, you should have known
\begin{theorem}
    Let $n$ be a positive integer, we have
    $$\frac{\mathrm d}{\mathrm dx}x^n=nx^{n-1}$$
\end{theorem}
by using induction.
There is another proof which is somewhat neater.
\begin{proof}
    First principle always does the trick.
    \begin{align*}
        &\lim_{h\to0}\frac{(x+h)^n-x^n}{h}\\
        =&\lim_{h\to0}\frac{1}{h}\sum_{m=1}^n\binom{n}{m}x^{n-m}h^m\\
        =&\lim_{h\to0}\sum_{m=1}^n\binom{n}{m}x^{n-m}h^{m-1}\\
        =&\binom{n}{1}x^{n-1}=nx^{n-1}
    \end{align*}
    As desired.
\end{proof}
Our existing knowledge about evaluating the derivative of sums then tells us that
\begin{theorem}
    $$\frac{\mathrm d}{\mathrm dx}\left(a_0+\sum_{i=1}^{n}a_ix^i\right)=\sum_{i=1}^{n}ia_ix^{i-1}$$
\end{theorem}
\begin{corollary}
    A non-constant polynomial's degree decreases by $1$ upon differentiation.
\end{corollary}
Obviously, by making use of the quotient rule, we can find our way through negative integer powers.
In fact, if $f(x)=x^{-n}$ where $n$ is a positive integer, we have
$$f^\prime(x)=\frac{0-nx^{n-1}}{x^{2n}}=(-n)x^{-n-1}$$
Combining with the fact that the constant function has a constantly zero derivative, we can see that
\begin{theorem}
    Let $f(x)=x^m$ where $m$ is any integer, then $f^\prime(x)=mx^{m-1}$.
\end{theorem}
In the next section, we shall show that the above theorem holds for any rational powers as well.